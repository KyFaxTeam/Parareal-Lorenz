\documentclass[11pt,a4paper]{article}

% Configuration de l'encodage et support du français
\usepackage[T1]{fontenc}
\usepackage[utf8]{inputenc}
\usepackage[french]{babel}
\usepackage{lmodern}  % Police avec un meilleur support des accents
\usepackage{eurosym}  % Pour le symbole euro et autres symboles européens

% Mise en page
\usepackage[top=2.5cm,bottom=2.5cm,left=2.5cm,right=2.5cm]{geometry}
\usepackage{fancyhdr}
\usepackage{titlesec}
\usepackage{titletoc}

% Mathématiques et sciences
\usepackage{amsmath}
\usepackage{amssymb}
\usepackage{physics}
\usepackage{siunitx}

% Graphiques et couleurs
\usepackage{tikz}
\usepackage{pgfplots}
\pgfplotsset{compat=newest}
\usepackage{xcolor}

% Tableaux et listes
\usepackage{booktabs}
\usepackage{array}
\usepackage{enumitem}
\usepackage{multirow}

% Images et figures
\usepackage{graphicx}
\usepackage{float}
\usepackage{wrapfig}
\usepackage{subcaption}

% Code et algorithmes
\usepackage{listings}
\usepackage{algorithm2e}
\usepackage[newfloat]{minted}

% Configuration de listings pour les accents
\lstset{
    inputencoding=utf8,
    extendedchars=true,
    literate=%
    {é}{{\'{e}}}1
    {è}{{\`{e}}}1
    {ê}{{\^{e}}}1
    {ë}{{\¨{e}}}1
    {à}{{\`{a}}}1
    {â}{{\^{a}}}1
    {î}{{\^{i}}}1
    {ï}{{\¨{i}}}1
    {ô}{{\^{o}}}1
    {ù}{{\`{u}}}1
    {û}{{\^{u}}}1
    {ü}{{\¨{u}}}1
    {ç}{{\c{c}}}1
}

% Références et bibliographie
\usepackage[hidelinks]{hyperref}
\usepackage{cleveref}
\usepackage[style=alphabetic,backend=biber]{biblatex}
\addbibresource{references.bib}

% Définition des couleurs personnalisées
\definecolor{primaryblue}{HTML}{1f77b4}
\definecolor{secondaryblue}{HTML}{7BA6C4}
\definecolor{accentorange}{HTML}{ff7f0e}
\definecolor{backgroundgray}{HTML}{f7f7f7}
\definecolor{tertiarygreen}{HTML}{2ca02c}

% Configuration des titres
\titleformat{\section}
{\color{primaryblue}\Large\bfseries}
{\thesection}{1em}{}[\titlerule]

\titleformat{\subsection}
{\color{secondaryblue}\large\bfseries}
{\thesubsection}{1em}{}

% Configuration des en-têtes et pieds de page
\pagestyle{fancy}
\fancyhf{}
\fancyhead[L]{\textit{Parallélisation temporelle du système de Lorenz}}
\fancyhead[R]{\thepage}
\fancyfoot[C]{\textit{\today}}

% Configuration des listings de code
\lstset{
    basicstyle=\ttfamily\small,
    breaklines=true,
    commentstyle=\color{green!60!black},
    keywordstyle=\color{blue},
    stringstyle=\color{red},
    numbers=left,
    numberstyle=\tiny\color{gray},
    frame=single,
    backgroundcolor=\color{backgroundgray},
    tabsize=4
}

% Paramètres de l'algorithme
\SetAlgoLined
\SetKwInput{KwInput}{Entrées}
\SetKwInput{KwOutput}{Sorties}

% Configuration de pgfplots
\pgfplotsset{
    every axis/.append style={
        line width=1pt,
        tick style={thin},
        grid style={thin,gray!30},
        grid=major,
        legend style={font=\small}
    }
}

% Titre du document
\title{%
    \textcolor{primaryblue}{\Huge\textbf{Parallélisation temporelle pour la résolution\\du système de Lorenz}}\\[1cm]
    \textcolor{secondaryblue}{\Large Une approche avec l'algorithme Parareal}\\[1cm]
    \vspace{0.5cm}
    \Large\textcolor{tertiarygreen}{
        Elonm AHOUANYE\\
        Yanel AÏNA\\[0.5cm]
        École Nationale Supérieure de Génie Mathématique et Modélisation\\
        (ENSGMM Abomey)\\[0.5cm]
        \large\textit{Supervisé par}\\
        \normalsize{Dr. DANDOGBESSI Bruno}\\
    }
}

\date{\today}

\begin{document}

\maketitle
\thispagestyle{empty}

\clearpage
\begin{abstract}
Ce document présente une étude approfondie de la parallélisation temporelle appliquée au système de Lorenz modifié pour une particule active guidée par sa mémoire. Nous explorons l'implémentation de l'algorithme Parareal comme alternative à la méthode RK4 séquentielle traditionnelle, en mettant l'accent sur les aspects théoriques et pratiques de cette approche novatrice.
\end{abstract}

\setcounter{tocdepth}{2}
\tableofcontents
\clearpage

% Introduction
% Introduction slides

\begin{frame}{Contexte physique : particules actives guidées par la mémoire}
    \begin{itemize}
        \item Les systèmes de particules actives : un domaine de la physique moderne
        \item Les gouttes "marcheuses" sur un bain liquide vibrant
        \item Interaction avec les ondes auto-générées
        \item Mémoire ondulatoire guidant le mouvement
    \end{itemize}
\end{frame}

\begin{frame}{Du système physique au modèle mathématique}
    \begin{itemize}
        \item Équation de trajectoire intégro-différentielle :
        \begin{equation}
            \ddot{x}_d + \dot{x}_d = F_{self} + F_{bias}
        \end{equation}
        \item Force du champ d'ondes auto-généré :
        \begin{equation}
            F_{self} = -R\int_{-\infty}^t W'(x_d(t)-x_d(s)) e^{-\frac{t-s}{\tau}} ds
        \end{equation}
    \end{itemize}
\end{frame}

\begin{frame}{Émergence du système de Lorenz}
    \begin{itemize}
        \item Simplification avec $W(x) = \cos(x)$
        \item Variables :
        \begin{align*}
            X &= \dot{x}_d \quad \text{(vitesse)} \\
            Y &= F_{self} \quad \text{(force de mémoire)} \\
            Z &= R\int_{-\infty}^t \cos(x_d(t)-x_d(s)) e^{-\frac{t-s}{\tau}} ds
        \end{align*}
        \item Système de Lorenz modifié :
        \begin{equation*}
        \begin{cases}
            \dot{X} = Y - X \\
            \dot{Y} = -\frac{1}{\tau} Y + XZ \\
            \dot{Z} = R - \frac{1}{\tau} Z - XY
        \end{cases}
        \end{equation*}
    \end{itemize}
\end{frame}

\begin{frame}{Problématique}
    \begin{itemize}
        \item \textbf{Défis de la résolution numérique :}
        \begin{itemize}
            \item Non-linéarité : termes de couplage $XZ$ et $XY$
            \item Sensibilité aux conditions initiales
            \item Échelles multiples : paramètre $\tau$
        \end{itemize}
        \vspace{0.5cm}
        \item \textbf{Objectifs :}
        \begin{itemize}
            \item Résolution séquentielle avec RK4
            \item Parallélisation temporelle avec l'algorithme Parareal
            \item Analyse des performances et de la précision
        \end{itemize}
    \end{itemize}
\end{frame}
\clearpage

% Resolution rk4 
% RK4 Method slides

\begin{frame}{Principe de la méthode RK4}
    \begin{itemize}
        \item Méthode classique de résolution numérique des EDO
        \item Approximation par combinaison de 4 évaluations :
        \begin{equation}
            u_{n+1} = u_n + \frac{h}{6}(k_1 + 2k_2 + 2k_3 + k_4)
        \end{equation}
        \item Coefficients $k_i$ : évaluations à différents points
        \begin{itemize}
            \item $k_1$ : pente initiale
            \item $k_2, k_3$ : pentes aux points milieu
            \item $k_4$ : pente finale
        \end{itemize}
    \end{itemize}
\end{frame}

\begin{frame}{Application au système de Lorenz}
    \begin{itemize}
        \item Pour un état $\mathbf{u} = (X, Y, Z)$ :
        \begin{equation*}
            \frac{d\mathbf{u}}{dt} = \mathbf{f}(\mathbf{u}) = \begin{pmatrix}
                Y - X \\
                -\frac{1}{\tau}Y + XZ \\
                R - \frac{1}{\tau}Z - XY
            \end{pmatrix}
        \end{equation*}
        \item Calcul des coefficients :
        \begin{align*}
            k_1 &= \mathbf{f}(t_n, \mathbf{u}_n) \\
            k_2 &= \mathbf{f}(t_n + \frac{h}{2}, \mathbf{u}_n + \frac{h}{2}k_1) \\
            k_3 &= \mathbf{f}(t_n + \frac{h}{2}, \mathbf{u}_n + \frac{h}{2}k_2) \\
            k_4 &= \mathbf{f}(t_n + h, \mathbf{u}_n + hk_3)
        \end{align*}
    \end{itemize}
\end{frame}

\begin{frame}{Paramètres de simulation}
    \begin{itemize}
        \item \textbf{Configuration standard :}
        \begin{itemize}
            \item Pas de temps : $h = 0.01$
            \item Durée totale : $T = 100.0$
            \item Amplitude des ondes : $R = 2.5$
            \item Conditions initiales : $(X_0, Y_0, Z_0) = (1.0, 0.0, 0.0)$
        \end{itemize}
        \vspace{0.3cm}
        \item \textbf{Régimes étudiés :}
        \begin{itemize}
            \item $\tau = 0.5$ : État Non-Marcheur
            \item $\tau = 2.0$ : Marche Régulière
            \item $\tau = 5.0$ : Marche Chaotique
        \end{itemize}
    \end{itemize}
\end{frame}

\begin{frame}{Limitations pour la parallélisation}
    \begin{itemize}
        \item \textbf{Double dépendance séquentielle :}
        \begin{enumerate}
            \item \textbf{Temporelle} : 
            \begin{equation*}
                u_{n+1} = \Phi_{\text{RK4}}(u_n)
            \end{equation*}
            \item \textbf{Interne} : calcul séquentiel des $k_i$
            \begin{equation*}
                k_i = f(k_1, \ldots, k_{i-1})
            \end{equation*}
        \end{enumerate}
        \vspace{0.3cm}
        \item \textbf{Conséquences :}
        \begin{itemize}
            \item Pas de calcul indépendant des étapes temporelles
            \item Impossibilité de parallélisation directe
        \end{itemize}
    \end{itemize}
\end{frame}
\clearpage

% Algorithme Parareal


\section{Algorithme Parareal}
\label{sec:algorithme_parareal}

\subsection{Principe fondamental}
L'algorithme Parareal introduit une approche qui brise la barrière de la séquentialité temporelle inhérente aux méthodes classiques. Le principe fondamental repose sur une décomposition du domaine temporel combinée à un processus itératif de correction.

\noindent Cette approche résout les limitations de RK4 en :
\begin{itemize}
    \item Permettant le calcul parallèle sur différents intervalles de temps
    \item Maintenant la précision grâce au propagateur fin
    \item Assurant la convergence via le processus itératif de correction
\end{itemize}

\subsection{Décomposition temporelle}
L'intervalle de temps global est divisé en $N$ sous-intervalles :
\begin{equation}
[0,T] = \bigcup_{i=0}^{N-1} [T_i,T_{i+1}]
\end{equation}

Cette décomposition permet une distribution naturelle du calcul sur plusieurs processeurs, chacun traitant un sous-intervalle spécifique.

\begin{figure}[h]
    \centering
    \begin{tikzpicture}[scale=1.0]
        % Axe temporel
        \draw[->] (0,0) -- (10,0) node[right] {$t$};
        \draw (0,-0.2) -- (0,0.2) node[above] {$T_0$};
        \draw (10,-0.2) -- (10,0.2) node[above] {$T_N$};
        
        % Subdivisions
        \foreach \x in {2,4,6,8} {
            \draw (\x,-0.2) -- (\x,0.2) node[above] {$T_{\x/2}$};
        }
        
        % Intervalles parallèles
        \foreach \x in {0,2,4,6,8} {
            \draw[blue, thick] (\x,1) -- (\x+2,1);
            \node[above] at (\x+1,1) {Proc. \number\numexpr\x/2+1};
        }
        
        % Points de synchronisation
        \foreach \x in {2,4,6,8} {
            \draw[red,dashed] (\x,0.5) -- (\x,1.5);
        }
    \end{tikzpicture}
    \caption{Décomposition temporelle et distribution sur les processeurs}
    \label{fig:decomposition}
\end{figure}


\subsection{Architecture à deux niveaux}
L'algorithme repose sur l'utilisation de deux propagateurs complémentaires :

\begin{enumerate}
    \item \textbf{Propagateur grossier $\mathcal{G}$} :
    \begin{itemize}
        \item Rapide mais approximatif
        \item Utilisé pour la prédiction initiale
        \item Typiquement basé sur une méthode d'Euler
    \end{itemize}
    
    \item \textbf{Propagateur fin $\mathcal{F}$} :
    \begin{itemize}
        \item Précis mais coûteux en calcul
        \item Appliqué en parallèle sur les sous-intervalles
        \item Basé sur RK4 dans notre implémentation
    \end{itemize}
\end{enumerate}

\subsection{Processus itératif}
L'algorithme procède en trois phases principales :

\subsubsection{Phase 1 : Initialisation}
\begin{enumerate}
    \item Division de $[0,T]$ en $N$ sous-intervalles
    \item Initialisation : $U_n^0 = u_0$ pour $n = 0$
    \item Prédiction grossière initiale :
    \begin{equation}
        U_{n+1}^0 = \mathcal{G}(T_n, U_n^0, \Delta T) \quad \text{pour } n = 0,\ldots,N-1
    \end{equation}
\end{enumerate}

\subsubsection{Phase 2 : Calcul parallèle}
Pour chaque itération $k$ :
\begin{enumerate}
    \item Calcul en parallèle sur chaque sous-intervalle :
    \begin{equation}
        \mathcal{F}(T_n, U_n^k, \Delta T) \quad \text{pour tout } n
    \end{equation}
    
    \item Application de la formule de correction :
    \begin{equation}
        U_{n+1}^{k+1} = \mathcal{G}(T_n, U_n^{k+1}, \Delta T) + \mathcal{F}(T_n, U_n^k, \Delta T) - \mathcal{G}(T_n, U_n^k, \Delta T)
        \label{eq:correction}
    \end{equation}
\end{enumerate}

\subsubsection{Phase 3 : Test de convergence}
\begin{itemize}
    \item Vérification du critère de convergence :
    \begin{equation}
        \max_n \|U_n^{k+1} - U_n^k\| < \varepsilon
    \end{equation}
    \item Si non convergé et $k < k_{max}$, retour à la Phase 2
\end{itemize}

\subsection{Analyse de la formule de correction}
La formule de correction (\ref{eq:correction}) peut être interprétée comme :
\begin{itemize}
    \item Une prédiction grossière de l'état suivant : $\mathcal{G}(T_n, U_n^{k+1}, \Delta T)$
    \item Une correction basée sur l'erreur du propagateur grossier :
    \begin{equation}
        \delta^k = \mathcal{F}(T_n, U_n^k, \Delta T) - \mathcal{G}(T_n, U_n^k, \Delta T)
    \end{equation}
\end{itemize}

\begin{figure}[h]
    \centering
    \begin{tikzpicture}[scale=1.0]
        % Axe du temps
        \draw[->] (0,0) -- (6,0) node[right] {$t$};
        
        % Solutions
        \draw[blue, thick] plot[domain=0:5,smooth] coordinates {
            (0,0) (1,0.8) (2,1.2) (3,1.4) (4,1.3) (5,1.5)
        } node[right] {Solution exacte};
        
        \draw[red, dashed] plot[domain=0:5,smooth] coordinates {
            (0,0) (1,0.7) (2,1.0) (3,1.2) (4,1.1) (5,1.3)
        } node[right] {$\mathcal{G}$};
        
        \draw[green!60!black, dotted, thick] plot[domain=0:5,smooth] coordinates {
            (0,0) (1,0.75) (2,1.15) (3,1.35) (4,1.25) (5,1.45)
        } node[right] {$\mathcal{F}$};
        
        % Points de correction
        \foreach \x/\y in {1/0.7, 2/1.0, 3/1.2, 4/1.1} {
            \draw[->] (\x,\y) -- (\x,{\y+0.1});
            \fill (\x,\y) circle (1.5pt);
        }
        
        \node at (3,-0.5) {Points de correction};
    \end{tikzpicture}
    \caption{Illustration du processus de correction}
    \label{fig:correction}
\end{figure}

\subsection{Propriétés de convergence}
La convergence de l'algorithme dépend de plusieurs facteurs :

\begin{enumerate}
    \item \textbf{Qualité du propagateur grossier} :
    \begin{itemize}
        \item Doit capturer suffisamment bien la dynamique du système
        \item Compromis entre précision et rapidité
    \end{itemize}
    
    \item \textbf{Taille des sous-intervalles} :
    \begin{itemize}
        \item Impact sur la vitesse de convergence
        \item Influence sur l'efficacité de la parallélisation
    \end{itemize}
    
    \item \textbf{Nature du système} :
    \begin{itemize}
        \item Sensibilité aux conditions initiales
        \item Non-linéarités et raideur
    \end{itemize}
\end{enumerate}

\subsection{Application au système de Lorenz}
Pour le système de Lorenz modifié étudié, l'application de l'algorithme Parareal nécessite une attention particulière à plusieurs aspects spécifiques :

\subsubsection{Adaptation des propagateurs}
\begin{enumerate}
    \item \textbf{Propagateur grossier $\mathcal{G}$} :
    \begin{itemize}
        \item Utilisation d'une méthode de RK2 :
        \begin{equation}
        \begin{cases}
            X_{n+1} = X_n + \Delta t(Y_n - X_n) \\
            Y_{n+1} = Y_n + \Delta t(-\frac{1}{\tau}Y_n + X_nZ_n) \\
            Z_{n+1} = Z_n + \Delta t(R - \frac{1}{\tau}Z_n - X_nY_n)
        \end{cases}
        \end{equation}
        \item Pas de temps adaptatif basé sur $\tau$ :
        \begin{equation}
        \Delta t_{\mathcal{G}} = \min(\alpha\tau, \Delta T)
        \end{equation}
        où $\alpha$ est un facteur de sécurité ($\approx 0.1$)
    \end{itemize}

    \item \textbf{Propagateur fin $\mathcal{F}$} :
    \begin{itemize}
        \item Méthode RK4 avec pas de temps fin :
        \begin{equation}
        \Delta t_{\mathcal{F}} = \frac{\Delta t_{\mathcal{G}}}{m}
        \end{equation}
        où $m$ est typiquement choisi entre 10 et 100
        \item Conservation des invariants du système
    \end{itemize}
\end{enumerate}

\subsubsection{Considérations dynamiques}
Les caractéristiques particulières du système influencent l'application de Parareal :

\begin{enumerate}
    \item \textbf{Régimes de comportement} :
    \begin{itemize}
        \item Régime transitoire : nécessite une plus grande précision du propagateur grossier
        \item Régime établi : permet des pas de temps plus grands
        \item Zones de bifurcation : requiert une attention particulière
    \end{itemize}

    \item \textbf{Échelles de temps} :
    \begin{figure}[h]
        \centering
        \begin{tikzpicture}[scale=0.8]
            % Axe temporel avec échelles
            \draw[->] (0,0) -- (10,0) node[right] {$t$};
            \draw (0,-0.2) -- (0,0.2);
            
            % Échelles caractéristiques
            \draw[blue,thick] (0,-1) -- (2,-1) node[right] {$\tau$ (mémoire)};
            \draw[red,thick] (0,-1.5) -- (0.5,-1.5) node[right] {$1/\omega$ (oscillations)};
            \draw[green!60!black,thick] (0,-2) -- (4,-2) node[right] {$T_c$ (temps caractéristique)};
            
            % Décomposition Parareal
            \foreach \x in {0,2,4,6,8} {
                \draw[gray,dashed] (\x,0.5) -- (\x,2.5);
                \node at (\x+1,2) {$\Delta T$};
            }
        \end{tikzpicture}
        \caption{Différentes échelles de temps du système}
        \label{fig:echelles_temps}
    \end{figure}
    
    \item \textbf{Impact du paramètre $\tau$} :
    \begin{itemize}
        \item Influence sur la taille optimale des sous-intervalles
        \item Ajustement de la fréquence des points de synchronisation
        \item Configuration du ratio $\Delta t_{\mathcal{F}}/\Delta t_{\mathcal{G}}$
    \end{itemize}
\end{enumerate}

\subsubsection{Stratégies de convergence}
Pour assurer une convergence efficace, plusieurs stratégies sont mises en place :

\begin{enumerate}
    \item \textbf{Critères adaptatifs} :
    \begin{equation}
    \varepsilon_k = \max\left\{\frac{\|U_n^{k+1} - U_n^k\|}{\|U_n^k\|}, \frac{|E_k - E_{k-1}|}{|E_k|}\right\} < \text{tol}
    \end{equation}
    où $E_k$ est l'énergie du système à l'itération $k$

    \item \textbf{Décomposition intelligente} :
    \begin{itemize}
        \item Intervalles plus courts dans les zones de forte non-linéarité
        \item Adaptation basée sur les estimateurs d'erreur locale
        \item Équilibrage de charge entre processeurs
    \end{itemize}

    \item \textbf{Gestion des instabilités} :
    \begin{itemize}
        \item Détection précoce des divergences
        \item Mécanismes de repli sur des pas plus petits
        \item Conservation des quantités physiques importantes
    \end{itemize}
\end{enumerate}

\subsubsection{Optimisations spécifiques}
Plusieurs optimisations sont implémentées pour améliorer l'efficacité :

\begin{itemize}
    \item \textbf{Prédiction améliorée} :
    \begin{equation}
    U_{n+1}^0 = \mathcal{G}(T_n, U_n^0, \Delta T) + \beta(U_n^0 - U_{n-1}^0)
    \end{equation}
    où $\beta$ est un facteur d'extrapolation basé sur la dynamique locale

    \item \textbf{Réutilisation des calculs} :
    \begin{itemize}
        \item Stockage intelligent des états intermédiaires
        \item Mise en cache des trajectoires partielles
        \item Réduction des communications MPI
    \end{itemize}

    \item \textbf{Adaptation dynamique} :
    \begin{itemize}
        \item Ajustement automatique des paramètres
        \item Modification de la décomposition en cours d'exécution
        \item Équilibrage de charge basé sur les performances observées
    \end{itemize}
\end{itemize}

Ces considérations spécifiques au système de Lorenz permettent d'optimiser l'efficacité de l'algorithme Parareal tout en maintenant la précision nécessaire pour capturer correctement la dynamique du système.
\clearpage

% Implémentation
\section{Implémentation}

\subsection{Architecture logicielle}
L'implémentation suit une architecture modulaire avec une séparation claire des responsabilités, organisée en trois niveaux principaux.

\begin{figure}[h]
    \centering
    \begin{tikzpicture}[
        block/.style={rectangle,draw,fill=primaryblue!20,minimum width=2.5cm,minimum height=1cm},
        subblock/.style={rectangle,draw,fill=primaryblue!10,minimum width=2cm,minimum height=0.8cm},
        arrow/.style={->,thick},
        ]
        
        % Niveau 1 : Programme principal
        \node[block] (main) at (0,2) {Programme principal (main.f90)};
        
        % Niveau 2 : Modules principaux
        \node[block] (solvers) at (-4,0) {Solveurs};
        \node[block] (parareal) at (0,0) {Parareal};
        \node[block] (domain) at (4,0) {Décomposition};
        
        % Niveau 3 : Sous-modules
        \node[subblock] (rk4) at (-6,-2) {RK4};
        \node[subblock] (ab) at (-4,-2) {AB2/AB3};
        \node[subblock] (euler) at (-2,-2) {Euler};
        \node[subblock] (mpi) at (0,-2) {MPI};
        \node[subblock] (conv) at (2,-2) {Convergence};
        \node[subblock] (deriv) at (4,-2) {Dérivées};
        
        % Connexions niveau 1-2
        \draw[arrow] (main) -- (solvers);
        \draw[arrow] (main) -- (parareal);
        \draw[arrow] (main) -- (domain);
        
        % Connexions niveau 2-3
        \draw[arrow] (solvers) -- (rk4);
        \draw[arrow] (solvers) -- (ab);
        \draw[arrow] (solvers) -- (euler);
        \draw[arrow] (parareal) -- (mpi);
        \draw[arrow] (parareal) -- (conv);
        \draw[arrow] (domain) -- (deriv);
        
        % Connexions horizontales
        \draw[arrow, dashed] (rk4) -- (mpi);
        \draw[arrow, dashed] (ab) -- (mpi);
        \draw[arrow, dashed] (euler) -- (mpi);
    \end{tikzpicture}
    \caption{Architecture détaillée du système}
    \label{fig:architecture_detailed}
\end{figure}

\begin{figure}[H]
    \centering
    \includegraphics[width=0.8\textwidth]{code/figure.png}
    \caption{Architecture logicielle : modules et sous-modules}
    \label{fig:architecture}
\end{figure}

\subsection{Composants principaux}

\subsubsection{Programme principal (main.f90)}
Point d'entrée centralisant :
\begin{itemize}
    \item Gestion des arguments et configurations
    \item Initialisation MPI et distribution des tâches
    \item Coordination des solveurs et mesure des performances
\end{itemize}

\subsubsection{Module de solveurs}
Implémente trois niveaux de solveurs :

\begin{enumerate}
    \item \textbf{Solveur fin (RK4)} :
    \begin{itemize}
        \item Haute précision pour les calculs critiques
        \item Adaptation automatique du pas de temps
        \item Détection et gestion des instabilités
    \end{itemize}

    \item \textbf{Solveurs intermédiaires (AB2/AB3)} :
    \begin{itemize}
        \item Compromis précision/performance
        \item Stabilisation par amortissement
        \item Mélange avec l'historique pour les régimes chaotiques
    \end{itemize}

    \item \textbf{Solveur grossier (Euler)} :
    \begin{itemize}
        \item Rapidité d'exécution pour les prédictions initiales
        \item Robustesse pour les grands pas de temps
        \item Adaptation aux différents régimes de $\tau$
    \end{itemize}
\end{enumerate}

\subsection{Mécanismes de stabilité et sécurité}

\subsubsection{Adaptation aux régimes dynamiques}
Le système ajuste automatiquement ses paramètres selon la valeur de $\tau$ :

\begin{lstlisting}[language=Fortran,caption=Adaptation dynamique des paramètres]
if (tau < 1.0) then  ! Régime non-marcheur
    h_coarse = min(h_coarse, tau/20.0)
    h_fine = min(h_fine, tau/200.0)
    adapt_tol = min(tol, 1.0E-5)
else if (tau < 3.0) then  ! Marche régulière
    h_coarse = min(h_coarse, tau/10.0)
    h_fine = min(h_fine, tau/100.0)
    adapt_tol = min(tol, 5.0E-6)
else  ! Régimes chaotiques
    h_coarse = min(h_coarse, 0.1)
    h_fine = min(h_fine, 0.01)
    adapt_tol = min(tol, 1.0E-6)
end if
\end{lstlisting}

\subsubsection{Mécanismes de protection}
Implémentation de plusieurs niveaux de sécurité :

\begin{lstlisting}[language=Fortran,caption=Circuit breaker pour les instabilités]
if (any(isnan(u)) .or. any(abs(u) > MAX_VALUE)) then
    bad_value_counter = bad_value_counter + 1
    where (isnan(u)) u = 0.0
    where (abs(u) > MAX_VALUE) 
        u = sign(MAX_VALUE, u)
    end where
    
    if (bad_value_counter >= MAX_BAD_ITER) then
        status = ERROR_UNSTABLE
        return
    end if
end if
\end{lstlisting}

\subsection{Parallélisation avancée}
\subsubsection{Distribution et équilibrage}
La décomposition temporelle utilise une stratégie adaptative :

\begin{lstlisting}[language=Fortran,caption=Distribution des intervalles]
subroutine distribute_intervals(total_intervals, size, rank, 
                              start_idx, end_idx)
    integer, intent(in) :: total_intervals, size, rank
    integer, intent(out) :: start_idx, end_idx
    integer :: base_count, remainder
    
    base_count = total_intervals / size
    remainder = mod(total_intervals, size)
    
    if (rank < remainder) then
        start_idx = rank * (base_count + 1) + 1
        end_idx = start_idx + base_count
    else
        start_idx = rank * base_count + remainder + 1
        end_idx = start_idx + base_count - 1
    end if
end subroutine distribute_intervals
\end{lstlisting}

\subsection{Visualisation et analyse}
Le système inclut des capacités avancées de visualisation :

\begin{itemize}
    \item Génération de trajectoires denses adaptées au régime :
    \begin{itemize}
        \item 50 points/intervalle pour $\tau$ < 2.0
        \item 75 points/intervalle pour 2.0 $\leq$ $\tau$ < 5.0
        \item 100 points/intervalle pour $\tau$ $\geq$ 5.0
    \end{itemize}
    \item Sauvegarde structurée des résultats pour post-traitement
    \item Scripts d'analyse automatisés pour validation
\end{itemize}

\subsection{Stratégies de convergence avancées}
L'algorithme implémente plusieurs stratégies sophistiquées pour assurer et accélérer la convergence :

\subsubsection{Critère de convergence adaptatif}
Le système utilise une métrique composite pour évaluer la convergence :

\begin{lstlisting}[language=Fortran,caption=Critère de convergence hybride]
! Changement d'état relatif
rel_state_change = max_diff / (maxval(abs(U_n)) + 1.0E-10)

! Conservation de l'énergie
rel_energy_change = abs(energy_k - energy_k_prev) 
                   / (abs(energy_k_prev) + 1.0E-10)

! Métrique combinée
conv_metric = max(rel_state_change, rel_energy_change)
converged = conv_metric < adapt_tol
\end{lstlisting}

\subsubsection{Prédiction améliorée}
Pour les régimes difficiles, une stratégie d'extrapolation est utilisée :

\begin{lstlisting}[language=Fortran,caption=Extrapolation pour prédiction]
if (k > 1) then
    ! Utilisation de l'historique des corrections
    U_new = propagate_with_ab3(...) + 
            beta * (U_n - U_prev)
    
    ! Stabilisation pour les petits tau
    if (tau < 1.0) then
        U_new = 0.8 * u_coarse_new + 
                0.2 * (u_fine - u_coarse_prev + u_coarse_new)
    end if
end if
\end{lstlisting}


L'ensemble du code implémenté est disponible en accès libre sur le dépôt GitHub suivant : https://github.com/KyFaxTeam/Parareal-Lorenz. 

\clearpage

% Résultats
\section{Résultats et analyse comparative}

\subsection{Méthodologie de comparaison}
Pour valider l'implémentation de l'algorithme Parareal et évaluer sa précision, nous avons effectué une comparaison systématique avec la méthode RK4 séquentielle. Pour chaque régime dynamique (caractérisé par différentes valeurs de $\tau$), nous avons analysé :
\begin{itemize}
    \item L'évolution temporelle des trois variables (X, Y, Z)
    \item L'erreur absolue entre les solutions Parareal et RK4 pour chacune des variables.
    \item La convergence vers l'état stationnaire ou l'attracteur chaotique
    \item Les portraits de phase dans les plans X-Z, Y-Z et X-Y
\end{itemize}

\subsection{Analyse par régime dynamique}

\subsubsection{Régime non-marcheur ($\tau$ = 0.5)}
Pour le régime de faible mémoire, où le système converge vers un point fixe :

\begin{figure}[H]
    \centering
    \includegraphics[width=\textwidth]{figures/comparisons/comparison_tau0.5_comparison}
    \caption{Comparaison des évolutions temporelles et erreurs temporelles pour $\tau$ = 0.5}
    \label{fig:comp_tau0.5_time}
\end{figure}

Les résultats montrent :
\begin{itemize}
    \item Une excellente concordance entre les deux méthodes dans la prédiction de la convergence vers l'origine
    \item Des erreurs absolues très faibles ($< 10^{-6}$) après la phase transitoire initiale
    \item Une stabilité numérique comparable pour les deux approches
\end{itemize}

\begin{figure}[H]
    \centering
    \includegraphics[width=\textwidth]{figures/comparisons/comparison_tau0.5_phase_portraits}
    \caption{Portraits de phase comparés pour $\tau$ = 0.5}
    \label{fig:comp_tau0.5_phase}
\end{figure}

Les portraits de phase confirment la précision de l'algorithme Parareal dans la capture de la dynamique de convergence.

\subsubsection{Régime de marche régulière ($\tau$ = 2.0)}
Pour le régime de marche stable :

\begin{figure}[H]
    \centering
    \includegraphics[width=\textwidth]{figures/comparisons/comparison_tau2.0_comparison}
    \caption{Comparaison des évolutions temporelles et erreurs absolues pour $\tau$ = 2.0}
    \label{fig:comp_tau2.0_time}
\end{figure}

L'analyse révèle :
\begin{itemize}
    \item Une reproduction fidèle des états stationnaires non-triviaux
    \item Une stabilisation rapide des erreurs à des niveaux très bas
    \item Une capacité à maintenir la précision sur de longues durées de simulation
\end{itemize}

\begin{figure}[H]
    \centering
    \includegraphics[width=\textwidth]{figures/comparisons/comparison_tau2.0_phase_portraits}
    \caption{Portraits de phase comparés pour $\tau$ = 2.0}
    \label{fig:comp_tau2.0_phase}
\end{figure}

Les trajectoires dans l'espace des phases sont pratiquement indiscernables entre les deux méthodes.

\subsubsection{Régime chaotique ($\tau$ = 5.0)}
Dans le régime de forte non-linéarité :

\begin{figure}[H]
    \centering
    \includegraphics[width=\textwidth]{figures/comparisons/comparison_tau5.0_comparison}
    \caption{Comparaison des évolutions temporelles et erreurs absolues pour $\tau$ = 5.0}
    \label{fig:comp_tau5.0_time}
\end{figure}

Observations principales :
\begin{itemize}
    \item Les trajectoires restent cohérentes malgré la nature chaotique du système
    \item Les erreurs absolues montrent des pics correspondant aux transitions dynamiques
    \item La structure globale de l'attracteur est préservée
\end{itemize}

\begin{figure}[H]
    \centering
    \includegraphics[width=\textwidth]{figures/comparisons/comparison_tau5.0_phase_portraits}
    \caption{Portraits de phase comparés pour $\tau$ = 5.0}
    \label{fig:comp_tau5.0_phase}
\end{figure}

Les portraits de phase démontrent la capacité de l'algorithme Parareal à reproduire la structure complexe de l'attracteur étrange.

\subsection{Synthèse des performances}

La comparaison systématique des résultats permet de conclure que :

\begin{itemize}
    % \item \textbf{Précision} :
    % \begin{itemize}
    \item Erreurs relatives maintenues sous $10^{-4}$ pour les régimes stables
    \item Reproduction fidèle des caractéristiques qualitatives dans les régimes chaotiques
    \item Conservation des invariants du système
    % \end{itemize}
    
    % \item \textbf{Stabilité} :
    % \begin{itemize}
    %     \item Aucune divergence observée, même dans les régimes fortement non-linéaires
    %     \item Robustesse face aux transitions dynamiques
    %     \item Maintien de la précision sur de longues durées de simulation
    % \end{itemize}
    
\end{itemize}

Ces résultats valident l'approche Parareal comme une alternative viable à RK4 pour la simulation du système de Lorenz, offrant un compromis optimal entre précision et performance grâce à la parallélisation temporelle.

\subsection{Analyse des performances}

\subsubsection{Performances temporelles}
\begin{itemize}
    \item \textbf{Accélération} : Gain significatif avec une réduction significative du temps de calcul
    \item \textbf{Efficacité} : Maintien d'une efficacité supérieure à 50\% même à grande échelle
    \item \textbf{Scalabilité} : Comportement quasi-linéaire jusqu'à 250000 itérations
\end{itemize}

\begin{figure}[h]
    \centering
    \begin{subfigure}[b]{0.48\textwidth}
        \includegraphics[width=\textwidth]{figures/benchmarks/execution_time_steps}
        \caption{Temps d'exécution par pas de temps}
        \label{fig:exec_time}
    \end{subfigure}
    \begin{subfigure}[b]{0.48\textwidth}
        \includegraphics[width=\textwidth]{figures/benchmarks/growth_trends}
        \caption{Tendances de croissance}
        \label{fig:growth_trends}
    \end{subfigure}
    \caption{Analyse des performances en temps de calcul}
    \label{fig:performance_analysis}
\end{figure}

\subsubsection{Stabilité et convergence}
La convergence a été évaluée selon plusieurs critères :

\begin{itemize}
    \item \textbf{Précision temporelle} : Erreur relative maintenue sous $10^{-6}$
    \item \textbf{Conservation des invariants} : Préservation des structures dynamiques
    \item \textbf{Robustesse} : Stabilité maintenue même en régime chaotique
\end{itemize}

% \subsection{Stabilité numérique}
\vskip 0.5cm
La stabilité de la solution a été évaluée en fonction de différents paramètres :

\begin{itemize}
    \item \textbf{Pas de temps} : Impact sur la précision et la stabilité
    \item \textbf{Nombre d'itérations} : Compromis entre convergence et temps de calcul
    \item \textbf{Tolérance} : Influence sur la qualité des résultats
\end{itemize}

% \begin{figure}[H]
%     \centering
%     \begin{tikzpicture}[scale=0.8]
%         \begin{axis}[
%             xlabel={$\Delta t$},
%             ylabel={Erreur relative},
%             grid=major,
%             legend pos=north west,
%             title={Analyse de stabilité}
%         ]
%             % Courbes d'erreur pour différentes méthodes
%             \addplot[color=primaryblue] coordinates {
%                 (0.01,1e-4) (0.02,2e-4) (0.04,8e-4) (0.08,3e-3) (0.16,1e-2)
%             };
%             \addlegendentry{Parareal}
            
%             \addplot[color=accentorange,dashed] coordinates {
%                 (0.01,1e-4) (0.02,4e-4) (0.04,1.6e-3) (0.08,6e-3) (0.16,2e-2)
%             };
%             \addlegendentry{RK4}
%         \end{axis}
%     \end{tikzpicture}
%     \caption{Analyse de l'erreur en fonction du pas de temps}
%     \label{fig:stability}
% \end{figure}

% \begin{figure}[h]
%     \centering
%     \includegraphics[width=0.8\textwidth]{figures/benchmarks/benchmark_results}
%     \caption{Résultats des tests de performance et stabilité}
%     \label{fig:benchmark_results}
% \end{figure}


% \subsubsection{Accélération (Speedup)}
% L'accélération obtenue avec différents nombres de processeurs montre l'efficacité de la parallélisation.

% \begin{figure}[h]
%     \centering
%     \begin{tikzpicture}[scale=0.8]
%         \begin{axis}[
%             xlabel={Nombre de processeurs},
%             ylabel={Speedup},
%             grid=major,
%             legend pos=north west,
%             title={Accélération en fonction du nombre de processeurs}
%         ]
%             % Courbe de speedup idéal
%             \addplot[color=gray,dashed] coordinates {
%                 (1,1) (2,2) (4,4) (8,8) (16,16)
%             };
%             \addlegendentry{Speedup idéal}
            
%             % Courbe de speedup réel
%             \addplot[color=primaryblue,thick,mark=*] coordinates {
%                 (1,1) (2,1.8) (4,3.2) (8,5.6) (16,8.4)
%             };
%             \addlegendentry{Speedup observé}
%         \end{axis}
%     \end{tikzpicture}
%     \caption{Analyse du speedup}
%     \label{fig:speedup}
% \end{figure}

% \subsubsection{Efficacité parallèle}
% L'efficacité parallèle montre comment l'accélération se compare au cas idéal.

% \begin{table}[h]
%     \centering
%     \begin{tabular}{@{}lcccc@{}}
%         \toprule
%         \textbf{Processeurs} & \textbf{Temps (s)} & \textbf{Speedup} & \textbf{Efficacité} & \textbf{Itérations} \\
%         \midrule
%         1  & 100.0 & 1.00 & 100\% & 1 \\
%         2  & 55.5  & 1.80 & 90\%  & 2 \\
%         4  & 31.2  & 3.20 & 80\%  & 3 \\
%         8  & 17.8  & 5.60 & 70\%  & 3 \\
%         16 & 11.9  & 8.40 & 52\%  & 4 \\
%         \bottomrule
%     \end{tabular}
%     \caption{Mesures de performance}
%     \label{tab:performance}
% \end{table}

% \subsection{Analyse de convergence}

% \subsubsection{Taux de convergence}
% La vitesse de convergence de l'algorithme Parareal dépend de plusieurs facteurs.

% \begin{figure}[h]
%     \centering
%     \begin{tikzpicture}[scale=0.8]
%         \begin{axis}[
%             xlabel={Itération},
%             ylabel={Erreur relative},
%             ymode=log,
%             grid=major,
%             legend pos=north east,
%             title={Convergence pour différentes valeurs de $\tau$}
%         ]
%             % Courbes de convergence pour différentes valeurs de tau
%             \addplot[color=primaryblue,mark=*] coordinates {
%                 (1,1e-1) (2,1e-2) (3,1e-3) (4,1e-4) (5,1e-5)
%             };
%             \addlegendentry{$\tau = 2.0$}
            
%             \addplot[color=accentorange,mark=square] coordinates {
%                 (1,1e-1) (2,5e-2) (3,2e-3) (4,5e-4) (5,2e-5)
%             };
%             \addlegendentry{$\tau = 5.0$}
            
%             \addplot[color=secondaryblue,mark=triangle] coordinates {
%                 (1,1e-1) (2,8e-2) (3,5e-3) (4,2e-3) (5,1e-3)
%             };
%             \addlegendentry{$\tau = 8.9$}
%         \end{axis}
%     \end{tikzpicture}
%     \caption{Convergence de l'algorithme pour différents paramètres}
%     \label{fig:convergence_rates}
% \end{figure}

% \subsection{Impact des paramètres}

% \subsubsection{Influence de la taille des sous-intervalles}
% La décomposition temporelle affecte directement les performances.

% \begin{figure}[h]
%     \centering
%     \begin{tikzpicture}[scale=0.8]
%         \begin{axis}[
%             xlabel={Nombre de sous-intervalles},
%             ylabel={Temps d'exécution (s)},
%             grid=major,
%             legend pos=north west
%         ]
%             % Courbes pour différentes configurations
%             \addplot[color=primaryblue,mark=*] coordinates {
%                 (4,80) (8,45) (16,28) (32,20) (64,18)
%             };
%             \addlegendentry{8 processeurs}
            
%             \addplot[color=accentorange,mark=square] coordinates {
%                 (4,40) (8,25) (16,18) (32,15) (64,14)
%             };
%             \addlegendentry{16 processeurs}
%         \end{axis}
%     \end{tikzpicture}
%     \caption{Impact du nombre de sous-intervalles}
%     \label{fig:interval_impact}
% \end{figure}




\clearpage

% Conclusion
\section{Conclusion et perspectives}

\subsection{Synthèse des résultats}

Cette étude a permis de démontrer l'efficacité de l'algorithme Parareal pour la parallélisation temporelle du système de Lorenz modifié. Les principaux résultats obtenus sont :

\begin{itemize}
    \item Une accélération significative du temps de calcul.
    \item Une préservation de la précision numérique comparable à la méthode RK4 séquentielle
    \item Une convergence robuste même dans les régimes chaotiques du système
\end{itemize}

\begin{figure}[h]
    \centering
    \begin{tikzpicture}
        % Définition des styles
        \tikzset{
            block/.style={rectangle, draw, fill=primaryblue!20, 
                         text width=2.5cm, text centered, minimum height=1cm},
            arrow/.style={->, thick, >=latex}
        }
        
        % Diagramme de synthèse
        \node[block] (parallelization) at (0,0) {Parallélisation temporelle};
        \node[block] (performance) at (-3,-2) {Performance};
        \node[block] (precision) at (0,-2) {Précision};
        \node[block] (scalability) at (3,-2) {Extensibilité};
        
        % Connexions
        \draw[arrow] (parallelization) -- (performance);
        \draw[arrow] (parallelization) -- (precision);
        \draw[arrow] (parallelization) -- (scalability);
        
        % Annotations
        \node[text width=2cm, align=center] at (-3,-3) {Speedup};
        \node[text width=2cm, align=center] at (0,-3) {Erreur < $10^{-4}$};
        \node[text width=2cm, align=center] at (3,-3) {Jusqu'à 6 processeurs dans notre cas};
    \end{tikzpicture}
    \caption{Synthèse des performances obtenues}
    \label{fig:synthesis}
\end{figure}

% \subsection{Contributions principales}

% Notre travail a apporté plusieurs contributions significatives :

% \begin{enumerate}
%     \item \textbf{Méthodologique}
%     \begin{itemize}
%         \item Développement d'une implémentation robuste de l'algorithme Parareal
%         \item Adaptation spécifique pour les systèmes chaotiques
%         \item Optimisation des communications MPI
%     \end{itemize}
    
%     \item \textbf{Technique}
%     \begin{itemize}
%         \item Framework modulaire et extensible
%         \item Outils d'analyse et de visualisation
%         \item Documentation détaillée du code
%     \end{itemize}
    
%     \item \textbf{Scientifique}
%     \begin{itemize}
%         \item Validation de l'approche pour le système de Lorenz
%         \item Analyse approfondie des performances
%         \item Identification des limites et contraintes
%     \end{itemize}
% \end{enumerate}

\subsection{Limitations actuelles}

Malgré les résultats prometteurs, certaines limitations ont été identifiées :

\begin{itemize}
    \item \textbf{Dépendance temporelle} : La méthode Parareal est sensible à la taille des intervalles
    \item \textbf{Divergence de la solution en utilisant certains solveurs grossiers} : La méthode d'Euler peut introduire des erreurs significatives qui ne peuvent pas être corrigées par le solveur fin. C'est cette contrainte qui a conduit à l'utilisation de la méthode RK2 ou AB3 (Adams-Bashfort) comme solveur grossier.
    \item \textbf{Scalabilité} : Nous n'avons pas pu étudié l'efficacité en augmentant le nombre de processseurs au delà de 6.
    \item \textbf{Régimes chaotiques} : Convergence plus lente dans certains régimes 
\end{itemize}

% \subsection{Perspectives futures}

% Plusieurs pistes d'amélioration et d'extension ont été identifiées :

% \begin{table}[h]
%     \centering
%     \begin{tabular}{@{}llc@{}}
%         \toprule
%         \textbf{Aspect} & \textbf{Amélioration proposée} & \textbf{Priorité} \\
%         \midrule
%         Algorithmique & Adaptation dynamique des sous-intervalles & Haute \\
%         Performance & Hybridation MPI/OpenMP & Moyenne \\
%         Précision & Solveurs d'ordre supérieur & Basse \\
%         Applicatif & Extension à d'autres systèmes & Moyenne \\
%         \bottomrule
%     \end{tabular}
%     \caption{Programme de développement futur}
%     \label{tab:future_work}
% \end{table}

% \subsubsection{Améliorations algorithmiques}

% \begin{itemize}
%     \item \textbf{Adaptation dynamique}
%     \begin{itemize}
%         \item Ajustement automatique des sous-intervalles
%         \item Critères de convergence adaptatifs
%         \item Équilibrage de charge intelligent
%     \end{itemize}
    
%     \item \textbf{Hybridation des méthodes}
%     \begin{itemize}
%         \item Combinaison avec la parallélisation spatiale
%         \item Intégration de techniques multi-grilles
%         \item Approches multi-niveaux
%     \end{itemize}
% \end{itemize}

% \subsection{Impact et applications}

% Les résultats de cette étude ouvrent la voie à plusieurs applications :

% \begin{itemize}
%     \item \textbf{Simulation numérique}
%     \begin{itemize}
%         \item Modélisation climatique
%         \item Dynamique des fluides
%         \item Systèmes complexes
%     \end{itemize}
    
%     \item \textbf{Applications industrielles}
%     \begin{itemize}
%         \item Optimisation de processus
%         \item Contrôle en temps réel
%         \item Prédiction de comportements
%     \end{itemize}
% \end{itemize}

% \subsection{Recommandations}

% Pour les développements futurs, nous recommandons :

% \begin{enumerate}
%     \item L'adoption d'une approche incrémentale pour les améliorations
%     \item La mise en place de benchmarks standardisés
%     \item Le développement d'outils d'analyse automatisés
%     \item La collaboration avec d'autres équipes de recherche
% \end{enumerate}

% Cette étude constitue une base solide pour de futures recherches dans le domaine de la parallélisation temporelle des systèmes dynamiques. Les résultats obtenus démontrent le potentiel de l'algorithme Parareal pour accélérer la simulation de systèmes chaotiques, ouvrant ainsi de nouvelles perspectives pour l'étude des systèmes complexes.
\clearpage

% Références
\printbibliography[title=Références]

\end{document}