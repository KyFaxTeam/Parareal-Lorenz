% Introduction slides

\begin{frame}{Contexte physique : particules actives guidées par la mémoire}
    \begin{itemize}
        \item Les systèmes de particules actives : un domaine de la physique moderne
        \item Les gouttes "marcheuses" sur un bain liquide vibrant
        \item Interaction avec les ondes auto-générées
        \item Mémoire ondulatoire guidant le mouvement
    \end{itemize}
\end{frame}

\begin{frame}{Du système physique au modèle mathématique}
    \begin{itemize}
        \item Équation de trajectoire intégro-différentielle :
        \begin{equation}
            \ddot{x}_d + \dot{x}_d = F_{self} + F_{bias}
        \end{equation}
        \item Force du champ d'ondes auto-généré :
        \begin{equation}
            F_{self} = -R\int_{-\infty}^t W'(x_d(t)-x_d(s)) e^{-\frac{t-s}{\tau}} ds
        \end{equation}
    \end{itemize}
\end{frame}

\begin{frame}{Émergence du système de Lorenz}
    \begin{itemize}
        \item Simplification avec $W(x) = \cos(x)$
        \item Variables :
        \begin{align*}
            X &= \dot{x}_d \quad \text{(vitesse)} \\
            Y &= F_{self} \quad \text{(force de mémoire)} \\
            Z &= R\int_{-\infty}^t \cos(x_d(t)-x_d(s)) e^{-\frac{t-s}{\tau}} ds
        \end{align*}
        \item Système de Lorenz modifié :
        \begin{equation*}
        \begin{cases}
            \dot{X} = Y - X \\
            \dot{Y} = -\frac{1}{\tau} Y + XZ \\
            \dot{Z} = R - \frac{1}{\tau} Z - XY
        \end{cases}
        \end{equation*}
    \end{itemize}
\end{frame}

\begin{frame}{Problématique}
    \begin{itemize}
        \item \textbf{Défis de la résolution numérique :}
        \begin{itemize}
            \item Non-linéarité : termes de couplage $XZ$ et $XY$
            \item Sensibilité aux conditions initiales
            \item Échelles multiples : paramètre $\tau$
        \end{itemize}
        \vspace{0.5cm}
        \item \textbf{Objectifs :}
        \begin{itemize}
            \item Résolution séquentielle avec RK4
            \item Parallélisation temporelle avec l'algorithme Parareal
            \item Analyse des performances et de la précision
        \end{itemize}
    \end{itemize}
\end{frame}