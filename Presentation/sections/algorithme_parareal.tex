% Parareal Algorithm slides

\begin{frame}{Principe fondamental}
    \begin{itemize}
        \item Solution à la barrière de séquentialité temporelle
        \item \textbf{Éléments clés :}
        \begin{itemize}
            \item Décomposition du domaine temporel
            \item Deux niveaux de propagateurs
            \item Processus itératif de correction
        \end{itemize}
        \item \textbf{Avantages :}
        \begin{itemize}
            \item Calcul parallèle sur différents intervalles
            \item Maintien de la précision
            % \item Convergence garantie
        \end{itemize}
    \end{itemize}
\end{frame}

\begin{frame}{Architecture à deux niveaux}
    \begin{itemize}
        \item \textbf{Propagateur grossier $\mathcal{G}$}
        \begin{itemize}
            \item Rapide mais approximatif
            \item Basé sur RK2 ou méthode d'Euler
            \item Prédiction initiale
        \end{itemize}
        \vspace{0.3cm}
        \item \textbf{Propagateur fin $\mathcal{F}$}
        \begin{itemize}
            \item Précis mais coûteux
            \item Basé sur RK4
            \item Exécution parallèle
        \end{itemize}
    \end{itemize}
\end{frame}

\begin{frame}{Processus itératif}
    \begin{enumerate}
        \item \textbf{Initialisation}
        \begin{itemize}
            \item Division de $[0,T]$ en $N$ sous-intervalles
            \item $U_n^0 = u_0$ pour $n = 0$
            \item Prédiction grossière : $U_{n+1}^0 = \mathcal{G}(T_n, U_n^0, \Delta T)$
        \end{itemize}
        \vspace{0.2cm}
        \item \textbf{Calcul parallèle}
        \begin{itemize}
            \item Calcul fin : $\mathcal{F}(T_n, U_n^k, \Delta T)$
            \item Formule de correction :
            \begin{equation*}
                U_{n+1}^{k+1} = \mathcal{G}(T_n, U_n^{k+1}, \Delta T) + \mathcal{F}(T_n, U_n^k, \Delta T) - \mathcal{G}(T_n, U_n^k, \Delta T)
            \end{equation*}
        \end{itemize}
    \end{enumerate}
\end{frame}

\begin{frame}{Application au système de Lorenz}
    \begin{itemize}
        \item \textbf{Adaptation des propagateurs}
        \begin{itemize}
            \item Propagateur grossier : RK2 avec pas adaptatif
            \item $\Delta t_{\mathcal{G}} = \min(\alpha\tau, \Delta T)$
            \item Propagateur fin : RK4 avec pas fin
            \item $\Delta t_{\mathcal{F}} = \frac{\Delta t_{\mathcal{G}}}{m}$
        \end{itemize}
        \vspace{0.3cm}
        \item \textbf{Considérations particulières}
        \begin{itemize}
            \item Gestion des différents régimes dynamiques
            \item Adaptation aux échelles de temps du système
            \item Conservation des invariants physiques
        \end{itemize}
    \end{itemize}
\end{frame}

\begin{frame}{Stratégies de convergence}
    \begin{itemize}
        \item \textbf{Critères adaptatifs}
        \begin{equation*}
            \varepsilon_k = \max\left\{\frac{\|U_n^{k+1} - U_n^k\|}{\|U_n^k\|}, \frac{|E_k - E_{k-1}|}{|E_k|}\right\} < \text{tol}
        \end{equation*}
        \item \textbf{Optimisations}
        \begin{itemize}
            \item Prédiction améliorée avec extrapolation
            \item Stockage intelligent des états intermédiaires
            \item Équilibrage de charge dynamique
        \end{itemize}
    \end{itemize}
\end{frame}