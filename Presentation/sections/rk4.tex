% RK4 Method slides

\begin{frame}{Principe de la méthode RK4}
    \begin{itemize}
        \item Méthode classique de résolution numérique des EDO
        \item Approximation par combinaison de 4 évaluations :
        \begin{equation}
            u_{n+1} = u_n + \frac{h}{6}(k_1 + 2k_2 + 2k_3 + k_4)
        \end{equation}
        \item Coefficients $k_i$ : évaluations à différents points
        \begin{itemize}
            \item $k_1$ : pente initiale
            \item $k_2, k_3$ : pentes aux points milieu
            \item $k_4$ : pente finale
        \end{itemize}
    \end{itemize}
\end{frame}

\begin{frame}{Application au système de Lorenz}
    \begin{itemize}
        \item Pour un état $\mathbf{u} = (X, Y, Z)$ :
        \begin{equation*}
            \frac{d\mathbf{u}}{dt} = \mathbf{f}(\mathbf{u}) = \begin{pmatrix}
                Y - X \\
                -\frac{1}{\tau}Y + XZ \\
                R - \frac{1}{\tau}Z - XY
            \end{pmatrix}
        \end{equation*}
        \item Calcul des coefficients :
        \begin{align*}
            k_1 &= \mathbf{f}(t_n, \mathbf{u}_n) \\
            k_2 &= \mathbf{f}(t_n + \frac{h}{2}, \mathbf{u}_n + \frac{h}{2}k_1) \\
            k_3 &= \mathbf{f}(t_n + \frac{h}{2}, \mathbf{u}_n + \frac{h}{2}k_2) \\
            k_4 &= \mathbf{f}(t_n + h, \mathbf{u}_n + hk_3)
        \end{align*}
    \end{itemize}
\end{frame}

\begin{frame}{Paramètres de simulation}
    \begin{itemize}
        \item \textbf{Configuration standard :}
        \begin{itemize}
            \item Pas de temps : $h = 0.01$
            \item Durée totale : $T = 100.0$
            \item Amplitude des ondes : $R = 2.5$
            \item Conditions initiales : $(X_0, Y_0, Z_0) = (1.0, 0.0, 0.0)$
        \end{itemize}
        \vspace{0.3cm}
        \item \textbf{Régimes étudiés :}
        \begin{itemize}
            \item $\tau = 0.5$ : État Non-Marcheur
            \item $\tau = 2.0$ : Marche Régulière
            \item $\tau = 5.0$ : Marche Chaotique
        \end{itemize}
    \end{itemize}
\end{frame}

\begin{frame}{Limitations pour la parallélisation}
    \begin{itemize}
        \item \textbf{Double dépendance séquentielle :}
        \begin{enumerate}
            \item \textbf{Temporelle} : 
            \begin{equation*}
                u_{n+1} = \Phi_{\text{RK4}}(u_n)
            \end{equation*}
            \item \textbf{Interne} : calcul séquentiel des $k_i$
            \begin{equation*}
                k_i = f(k_1, \ldots, k_{i-1})
            \end{equation*}
        \end{enumerate}
        \vspace{0.3cm}
        \item \textbf{Conséquences :}
        \begin{itemize}
            \item Pas de calcul indépendant des étapes temporelles
            \item Impossibilité de parallélisation directe
        \end{itemize}
    \end{itemize}
\end{frame}