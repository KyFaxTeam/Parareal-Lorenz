% Option 1: Theme Berlin - Professional and modern
% \usetheme{Berlin}
% \usecolortheme{dolphin}
% \useinnertheme{circles}

% Option 2: Theme Singapore - Clean and minimalist
% \usetheme{Singapore}
% \usecolortheme{orchid}
% \useinnertheme{rectangles}

% Option 3: Theme Copenhagen - Academic and elegant
% \usetheme{Copenhagen}
% \usecolortheme{seahorse}
% \useoutertheme{infolines}

% Option 4: Theme Dresden - Classic with sidebar
\usetheme{Dresden}
\usecolortheme{default}
\useoutertheme{sidebar}

% Personnalisation pour arrondir les bords des barres
\setbeamertemplate{blocks}[rounded][shadow=true]
\makeatletter
% Arrondir les bords des barres
\setbeamertemplate{sidebar canvas left}[default]
\setbeamercolor{block title}{bg=primaryblue!80!white,fg=white}
\setbeamercolor{block body}{bg=primaryblue!10!white}
\setbeamercolor{block alerted title}{bg=accentorange!80!white,fg=white}
% \setbeamercolor{structure}{fg=secondaryblue} % Using the lighter blue defined in your theme

% Option 5: Theme Luebeck - Modern with header
% \usetheme{Luebeck}
% \usecolortheme{rose}
% \useoutertheme{miniframes}

% Pour utiliser, décommentez le thème souhaité dans main.tex
% et commentez les autres thèmes.

% Personnalisation commune
\setbeamertemplate{navigation symbols}{}
\setbeamertemplate{footline}[frame number]
\setbeamertemplate{caption}[numbered]
\setbeamertemplate{section in toc}[sections numbered]
\setbeamertemplate{subsection in toc}[subsections numbered]

% Couleurs personnalisées (adaptez selon le thème choisi)
\definecolor{primaryblue}{HTML}{1f77b4}
\definecolor{secondaryblue}{HTML}{7BA6C4}
\definecolor{accentorange}{HTML}{ff7f0e}
\definecolor{tertiarygreen}{HTML}{2ca02c}
\definecolor{backgroundgray}{HTML}{f7f7f7}

% Recommendations par thème:
%
% Berlin + dolphin:
% - Professionnel et moderne
% - Excellent pour les présentations techniques
% - Navigation claire avec la barre du haut
%
% Singapore + orchid:
% - Minimaliste et élégant
% - Met l'accent sur le contenu
% - Parfait pour les présentations scientifiques
%
% Copenhagen + seahorse:
% - Style académique traditionnel
% - Bonne hiérarchie visuelle
% - Navigation simple et efficace
%
% Dresden + beaver:
% - Sidebar pour navigation rapide
% - Bon pour les longues présentations
% - Structure claire avec sections/sous-sections
%
% Luebeck + rose:
% - Design moderne avec en-tête
% - Excellent contraste
% - Navigation intuitive